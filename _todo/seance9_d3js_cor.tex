%!TEX encoding =  IsoLatin
\input{../../common/exo_begin.tex}%
\firstpassagedo{\refstepcounter{cxtdcor}\refstepcounter{cxtdcor}\refstepcounter{cxtdcor}\refstepcounter{cxtdcor}
					\refstepcounter{cxtdcor}\refstepcounter{cxtdcor}\refstepcounter{cxtdcor}\refstepcounter{cxtdcor}}



\begin{xtdcor}{seance9_2013_d3js}\label{seance9_2013_d3js_cor}

\textbf{Exercice 1}

Utilisation d'OpenStreetMap pour visualiser des données~:

\inputcode{../python_td_2013/programme/seance9_d3js.py}{Utilisation d'OpenStreetMap pour visualiser des données}


Les stations Vélib dans les zones de travail~:

\inputcode{../python_td_2013/programme/seance9_d3js.py}{Les stations Vélib dans les zones de travail}

\partietdb{d3.js\footnote{\httpstyle{http://d3js.org/}}}


\textbf{Exercice 2}

\inputcode{../python_td_2013/programme/td9_graph_lworld.html}{Le réseau L-World}


\partietdc{Un graphique avec des événements}

\textbf{Exercice 2}

\inputcode{../python_td_2013/programme/td9_graph_lworld_event.html}{Le réseau L-World avec des événements} 


\partietdd{Zoomer}

La correction est en deux parties~:

\inputcode{../python_td_2013/programme/td9_by_hours.html}{Le fichier HTML contenant le code javascript.}

Un extrait des données contenant dans le fichier \codes{td9\_by\_hours\_data.js}~:

\inputcode{../python_td_2013/programme/td9_by_hours_data_small.js}{données incluses dans un script javascript}


\end{xtdcor}





\input{../../common/exo_end.tex}%
