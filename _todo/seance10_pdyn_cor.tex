%!TEX encoding =  IsoLatin
\input{../../common/exo_begin.tex}%
\firstpassagedo{\refstepcounter{cxtdcor}\refstepcounter{cxtdcor}\refstepcounter{cxtdcor}\refstepcounter{cxtdcor}
					\refstepcounter{cxtdcor}\refstepcounter{cxtdcor}\refstepcounter{cxtdcor}\refstepcounter{cxtdcor}\refstepcounter{cxtdcor}}



\begin{xtdcor}{seance10_2013_pdyn}\label{seance10_2013_pdyn_cor}

\textbf{Exercice 1 }

%%% r�cup�ration de la liste des villes (sans doublons) %%%
\begin{verbatimx}
vil = { }
for row in df.values :
    vil [row[0]] = 0
    vil [row[1]] = 1
vil = list(vil.keys())
print (len(vil))
\end{verbatimx}

\textbf{Exercice 2 }

%%% construction d'un dictionnaire { (a,b):d, (b,a):d } %%%
\begin{verbatimx}
dist = { }
for row in df.values :
    a = row[0]
    b = row[1]
    dist[a,b] = dist[b,a] = row[2]
print (len(dist))
\end{verbatimx}



%%% distance Charleville-Mezieres --> Bordeaux %%%
\begin{verbatimx}
print ( dist["Charleville-Mezieres","Bordeaux"] )
\end{verbatimx}


\partietdb{L'algorithme du plus court chemin (Dikjstra)}


\textbf{Exercice 3}

On peut remplir facilement toutes les cases correspondant aux villes reli�es � Charleville-Mezieres.

%%% initialisation du tableau d %%%
\begin{verbatimx}
d = { }
d['Charleville-Mezieres'] = 0
for v in vil : d[v] = 1e10
for v,w in dist :
    if v == 'Charleville-Mezieres': 
        d[w] = dist[v,w]
\end{verbatimx}


\textbf{Exercice 4}

Si on d�couvre que $d[w] > d[v] + dist[w,v]$, cela veut dire qu'il faut mettre � jour le tableau $d$ car il ne contient pas la distance optimale. On passe en revue toutes les distance connue~:

%%% une it�ration de l'algorithme de Dikjstra %%%
\begin{verbatimx}
for v,w in dist :
    d2 = d[v] + dist[v,w]
    if d2 < d[w] :
        d[w] = d2
\end{verbatimx}

On trouve 829140 m�tres pour la distance (Charleville-Mezieres, Bordeaux). Ce n'est pas forc�ment la meilleure. Pour �tre s�r, il faut r�p�ter la m�me it�ration autant de fois qu'il y a de villes (car le plus long chemin contient autant d'�tapes qu'il y a de villes). 

%%% l'algorithme de Dikjstra %%%
\begin{verbatimx}
for i in range(0,len(d)) :        
    for v,w in dist :
        d2 = d[v] + dist[v,w]
        if d2 < d[w] :
            d[w] = d2
\end{verbatimx}

On trouve 795670 m�tres.

        
\partietdc{La r�partition des skis}

\textbf{Exercice facultatif}

Pour montrer que l'algorithme sugg�r� permettra d'obtenir la solution optimale, il faut montrer qu'il n'est pas n�cessaire d'envisager aucun autre ordre que celui des skieurs et des paires tri�s par taille croissante. On consid�re donc un appariement $\sigma$ qui associ� le skieur $t_i$ � la paire $s_{\sigma(i)}$. Il suffit que montrer que~:

$$
\forall i,j,  \; t_i \infegal t_j \Longleftrightarrow s_{\sigma(i)} \infegal s_{\sigma(j)}
$$

Pour montrer cela, on fait un raisonnement par l'absurde~: pour $i$ et $j$ quelconques, on suppose qu'il existe un appariement optimal tel que  $t_i >  t_j$ et $s_{\sigma(i)} \infegal s_{\sigma(j)}$. Le co�t $C(\sigma)$ de cet appariement est~:

\begin{eqnarray}
C(\sigma) &=& \sum_{k=1}^{N} \abs{ t_k - s_{\sigma(k)} } = \alpha + \abs{ t_i - s_{\sigma(i)} } + \abs{ t_j - s_{\sigma(j)} } 
\end{eqnarray}

Le co�t de l'appariement en permutant les skieurs $i$ et $j$ (donc en les rangeant dans l'ordre croissant) est~:

\begin{eqnarray}
C(\sigma') &=& \sum_{k=1}^{N} \abs{ t_k - s_{\sigma(k)} } = \alpha + \abs{ t_j - s_{\sigma(i)} } + \abs{ t_i - s_{\sigma(j)} } 
\end{eqnarray}

On calcule~:

\begin{eqnarray}
C(\sigma) - C(\sigma') 
				&=& \abs{ t_i - s_{\sigma(i)} } + \abs{ t_j - s_{\sigma(j)} }  - \abs{ t_j - s_{\sigma(i)} } - \abs{ t_i - s_{\sigma(j)} } 
\end{eqnarray}

Premier cas~: $t_j \supegal s_{\sigma(i)} $ et $t_i > t_j \supegal s_{\sigma(i)}$

\begin{eqnarray}
C(\sigma) - C(\sigma') 
				&=& \abs{ t_i - s_{\sigma(i)} } + \abs{ t_j - s_{\sigma(j)} }  - \pa{  t_j - s_{\sigma(j)} + s_{\sigma(j)} - s_{\sigma(i)}} - \abs{ t_i - s_{\sigma(j)} } \\
				&=&  t_i - s_{\sigma(i)}  + \abs{ t_j - s_{\sigma(j)} }  - \pa{  t_j - s_{\sigma(j)}} -\pa{ s_{\sigma(j)} - s_{\sigma(i)}} - \abs{ t_i - s_{\sigma(j)} } \\
				&=&  t_i - s_{\sigma(i)}- \abs{ t_i - s_{\sigma(i)}}  + \abs{ t_j - s_{\sigma(j)} }  - \pa{  t_j - s_{\sigma(j)}} \\
				&=&  \abs{ t_j - s_{\sigma(j)} }  - \pa{  t_j - s_{\sigma(j)}} \\
				&\supegal& 0
\end{eqnarray}

Second cas~: $t_j \infegal s_{\sigma(i)} $ et $t_j \infegal s_{\sigma(i)} \infegal s_{\sigma(j)}$

\begin{eqnarray}
C(\sigma) - C(\sigma') 
				&=& \abs{ t_i - s_{\sigma(i)} } + s_{\sigma(j)} -t_j  - \pa{  s_{\sigma(i)} - t_j} - \abs{ t_i - s_{\sigma(j)} } \\
				&=& \abs{ t_i - s_{\sigma(i)} } + s_{\sigma(j)} -  s_{\sigma(i)} - \abs{ t_i - s_{\sigma(j)} } \\
				&\supegal& \abs{ t_i - s_{\sigma(j)}} - \abs{  s_{\sigma(j)} -  s_{\sigma(i)} } + s_{\sigma(j)} -  s_{\sigma(i)} - \abs{ t_i - s_{\sigma(j)} } \\
				&\supegal& 0
\end{eqnarray}

Dans les deux cas, on montre donc qu'il existe un meilleur appariement global en permutant les deux skieurs~$i$ et~$j$, c'est-�-dire en les triant par ordre croissant de taille. Nous avons donc montr� que, si les paires de ski sont tri�es par ordre croissant de taille, l'appariement optimal est n�cessaire un appariement pour lequel les skieurs sont aussi tri�s par ordre croissant. Lors de la recherche de cet appariement optimal, on peut se restreindre � ces cas de figure.

				
\textbf{Exercice 5}

$$
p(n,m) = \min \acc { p(n-1,m-1) + \abs{ t_n - s_m }, p(n,m-1) }
$$

Lorsqu'on consid�re le meilleur appariement des paires $1..m$ et des skieurs $1..n$, il n'y a que deux choix possibles pour la paire~$m$~:
\begin{enumerate}
\item soit elle n'est associ�e � aucun skieur et dans ce cas~: $p(n,m) = p(n,m-1)$,
\item soit elle est associ�e au skieur $n$ (et � aucun autre)~: $p(n,m) = p(n-1,m-1) + \abs{ t_n - s_m }$.
\end{enumerate}


\partietdd{La r�partition des skis (suite et fin)}

\textbf{Exercice 6}

%%% co�t de la meilleure distribution des skis  %%%
\begin{verbatimx}
p = { }
p [-1,-1] = 0
for n,taille in enumerate(skieurs) : p[n,-1] = p[n-1,-1] + taille
for m,paire  in enumerate(paires ) : p[-1,m] = 0
for n,taille in enumerate(skieurs) :
    for m,paire in enumerate(paires) :
        p1 = p.get ( (n  ,m-1), 1e10 ) 
        p2 = p.get ( (n-1,m-1), 1e10 ) + abs(taille - paire)
        p[n,m] = min(p1,p2)
        
print (p[len(skieurs)-1,len(paires)-1])
\end{verbatimx}

\textbf{Exercice 7}

Il faut imaginer que $p$ peut �tre repr�sent� sous forme de matrice et qu'� chaque fois, on prend le meilleur chemin parmi 2~:

$$
\xymatrix{  & m-1  & m \\
     n-1   &  p(n-1,m-1) \ar[dr]^{ + \abs{t_n - s_m}} &  p(n-1,m) \\
     n     &  p(n,m-1)   \ar[r] & p(n,m) \\
				}
$$

\begin{itemize}
\item Chemin horizontal~: on ne choisit pas la paire $m$.
\item Chemin diagonal~: on choisit la paire $m$ pour le skieur $n$.
\end{itemize}

Pour retrouver la distribution, il suffit de conserver � chaque �tape le meilleur des deux chemins.



%%% r�cup�ration de la meilleure distribution %%%
\begin{verbatimx}
p = { }
p [-1,-1] = 0
best = { }
for n,taille in enumerate(skieurs) : p[n,-1] = p[n-1,-1] + taille
for m,paire  in enumerate(paires ) : p[-1,m] = 0
for n,taille in enumerate(skieurs) :
    for m,paire in enumerate(paires) :
        p1 = p.get ( (n  ,m-1), 1e10 ) 
        p2 = p.get ( (n-1,m-1), 1e10 ) + abs(taille - paire)
        p[n,m] = min(p1,p2)
        
        if p[n,m] == p1 : best [n,m] = n,m-1
        else : best [n,m] = n-1,m-1
        
print (p[len(skieurs)-1,len(paires)-1])

chemin = [ ]
pos = len(skieurs)-1,len(paires)-1
while pos in best :
    print (pos)
    chemin.append(pos)
    pos = best[pos]
chemin.reverse()
print (chemin)
\end{verbatimx}




\partietdEND


Les deux algorithmes ont un co�t quadratique.






\end{xtdcor}





\input{../../common/exo_end.tex}%
