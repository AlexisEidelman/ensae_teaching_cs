%!TEX encoding =  IsoLatin
\input{../../common/exo_begin.tex}%
\firstpassagedo{\refstepcounter{cxtdcor}\refstepcounter{cxtdcor}\refstepcounter{cxtdcor}\refstepcounter{cxtdcor}
					\refstepcounter{cxtdcor}\refstepcounter{cxtdcor}\refstepcounter{cxtdcor}}



\begin{xtdcor}{seance8_2013_sql}\label{seance8_2013_sql_cor}

\partietda{R�cup�rer les donn�es}

\textbf{Exercice 1}

%%% compter le nombre de dates distinctes %%%
\begin{verbatimx}
SELECT COUNT(*) FROM (
    SELECT DISTINCT last_update FROM td8_velib
) ;
\end{verbatimx}
 

%%% premi�re et derni�re date %%%
\begin{verbatimx}
SELECT MIN(last_update), MAX(last_update) FROM td8_velib ;
\end{verbatimx}


\partietdb{GROUP BY}


%%% exercice 2 %%%
\begin{verbatimx}
SELECT number, COUNT(*) AS nb 
FROM td8_velib
WHERE available_bikes==0 AND last_update >= '2013-09-10 11:30:19'
GROUP BY number
ORDER BY nb DESC
\end{verbatimx}

%%% exercice 3 : plage horaires de cinq minutes o� il n'y a aucun v�lo disponible %%%
\begin{verbatimx}
SELECT nb, COUNT(*) AS nb_station
FROM (
  -- requ�te de l'exercice pr�c�dent
  SELECT number, COUNT(*) AS nb 
  FROM td8_velib
  WHERE available_bikes==0 AND last_update >= '2013-09-10 11:30:19'
  GROUP BY number
)
GROUP BY nb
\end{verbatimx}


\partietdc{JOIN}

%%% exercice 4 : distribution horaire par station et par tranche de 5 minutes %%%
\begin{verbatimx}
SELECT A.number, A.heure, A.minute, 1.0 * A.nb_velo / B.nb_velo_tot  AS distribution_temporelle
FROM (
  SELECT number, heure, minute, SUM(available_bikes) AS nb_velo
  FROM td8_velib
  WHERE last_update >= '2013-09-10 11:30:19'
  GROUP BY heure, minute, number
) AS A
JOIN (
  SELECT number, heure, minute, SUM(available_bikes) AS nb_velo_tot
  FROM td8_velib
  WHERE last_update >= '2013-09-10 11:30:19'
  GROUP BY number
) AS B
ON A.number == B.number
--WHERE A.number in (8001, 8003, 15024, 15031)  -- pour n'afficher que quelques stations
ORDER BY A.number, A.heure, A.minute
\end{verbatimx}


\partietdd{zones de travail et zones de r�sidences}



%%% Zone de travail et zone de r�sidence %%%
\begin{verbatimx}
SELECT number, SUM(distribution_temporelle) AS velo_jour
FROM (
  -- requ�te de l'exercice 4
	SELECT A.number, A.heure, A.minute, 1.0 * A.nb_velo / B.nb_velo_tot  AS distribution_temporelle
	FROM (
		SELECT number, heure, minute, SUM(available_bikes) AS nb_velo
		FROM td8_velib
		WHERE last_update >= '2013-09-10 11:30:19'
		GROUP BY heure, minute, number
	) AS A
	JOIN (
		SELECT number, heure, minute, SUM(available_bikes) AS nb_velo_tot
		FROM td8_velib
		WHERE last_update >= '2013-09-10 11:30:19'
		GROUP BY number
	) AS B
	ON A.number == B.number
)
WHERE heure >= 10 AND heure <= 16 
GROUP BY number
\end{verbatimx}




\partietdEND


On trouve les arrondissements o� les stations de v�lib sont les plus remplies en journ�e au centre de Paris.

%%% JOIN avec la table stations et les stations "travail" %%%
\begin{verbatimx}
SELECT C.number, name, lat, lng, velo_jour FROM 
(
  -- requ�te de la partie pr�c�dente
	SELECT number, SUM(distribution_temporelle) AS velo_jour
	FROM (
		-- requ�te de l'exercice 4
		SELECT A.number, A.heure, A.minute, 1.0 * A.nb_velo / B.nb_velo_tot  AS distribution_temporelle
		FROM (
			SELECT number, heure, minute, SUM(available_bikes) AS nb_velo
			FROM td8_velib
			WHERE last_update >= '2013-09-10 11:30:19'
			GROUP BY heure, minute, number
		) AS A
		JOIN (
			SELECT number, heure, minute, SUM(available_bikes) AS nb_velo_tot
			FROM td8_velib
			WHERE last_update >= '2013-09-10 11:30:19'
			GROUP BY number
		) AS B
		ON A.number == B.number
	)
	WHERE heure >= 10 AND heure <= 16 
	GROUP BY number
) AS C
INNER JOIN stations
ON C.number == stations.number 
\end{verbatimx}




\end{xtdcor}





\input{../../common/exo_end.tex}%
